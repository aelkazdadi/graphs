\documentclass{article}
\usepackage[utf8]{inputenc}


\begin{document}
\section*{Practical 1}%
\subsection*{Exercise 2}%
\label{sub:tp1ex2}
The size of the graph is calculated by iterating over the text file and
counting the number of lines containing pairs of integers.
Note that this is to be used on raw graphs, as cleaned ones already contain the
number of nodes and edges on the first line.
As a result of this and the fact that we avoid storing the edges in memory, the
program doesn't account for potential duplicates.

\subsection*{Exercise 3}%
\label{sub:tp1ex3}
The program first counts the number of lines using ``wc -l", to tell how much
memory it should allocate for the edge list, then we iterate over each line
and add the edges to the list in the correct order.
(tail is strictly smaller than head)\\
We then sort the edge list in lexicographical order.
(the order is based on tail of edge, and in case of equality, on the head of the
edge)\\
We iterate over the edge list to count the number of non duplicate edges,
and write the number of nodes and number of edges to a new text file.\\
We iterate once again on the list of edges and save them to a new text file,
skipping over duplicates. Note that node indices are shifted so that they start
from 0.

\subsection*{Exercise 4}%
\label{sub:tp1ex4}
We read the number of nodes and the number of edges from the cleaned files, and
allocate memory for a list of node degrees
We iterate over the (cleaned) file of list of edges, and for each edge, we
increment the degrees of its tail and its head.


\end{document}
